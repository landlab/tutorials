
% Default to the notebook output style

    


% Inherit from the specified cell style.




    
\documentclass[11pt]{article}

    
    
    \usepackage[T1]{fontenc}
    % Nicer default font (+ math font) than Computer Modern for most use cases
    \usepackage{mathpazo}

    % Basic figure setup, for now with no caption control since it's done
    % automatically by Pandoc (which extracts ![](path) syntax from Markdown).
    \usepackage{graphicx}
    % We will generate all images so they have a width \maxwidth. This means
    % that they will get their normal width if they fit onto the page, but
    % are scaled down if they would overflow the margins.
    \makeatletter
    \def\maxwidth{\ifdim\Gin@nat@width>\linewidth\linewidth
    \else\Gin@nat@width\fi}
    \makeatother
    \let\Oldincludegraphics\includegraphics
    % Set max figure width to be 80% of text width, for now hardcoded.
    \renewcommand{\includegraphics}[1]{\Oldincludegraphics[width=.8\maxwidth]{#1}}
    % Ensure that by default, figures have no caption (until we provide a
    % proper Figure object with a Caption API and a way to capture that
    % in the conversion process - todo).
    \usepackage{caption}
    \DeclareCaptionLabelFormat{nolabel}{}
    \captionsetup{labelformat=nolabel}

    \usepackage{adjustbox} % Used to constrain images to a maximum size 
    \usepackage{xcolor} % Allow colors to be defined
    \usepackage{enumerate} % Needed for markdown enumerations to work
    \usepackage{geometry} % Used to adjust the document margins
    \usepackage{amsmath} % Equations
    \usepackage{amssymb} % Equations
    \usepackage{textcomp} % defines textquotesingle
    % Hack from http://tex.stackexchange.com/a/47451/13684:
    \AtBeginDocument{%
        \def\PYZsq{\textquotesingle}% Upright quotes in Pygmentized code
    }
    \usepackage{upquote} % Upright quotes for verbatim code
    \usepackage{eurosym} % defines \euro
    \usepackage[mathletters]{ucs} % Extended unicode (utf-8) support
    \usepackage[utf8x]{inputenc} % Allow utf-8 characters in the tex document
    \usepackage{fancyvrb} % verbatim replacement that allows latex
    \usepackage{grffile} % extends the file name processing of package graphics 
                         % to support a larger range 
    % The hyperref package gives us a pdf with properly built
    % internal navigation ('pdf bookmarks' for the table of contents,
    % internal cross-reference links, web links for URLs, etc.)
    \usepackage{hyperref}
    \usepackage{longtable} % longtable support required by pandoc >1.10
    \usepackage{booktabs}  % table support for pandoc > 1.12.2
    \usepackage[inline]{enumitem} % IRkernel/repr support (it uses the enumerate* environment)
    \usepackage[normalem]{ulem} % ulem is needed to support strikethroughs (\sout)
                                % normalem makes italics be italics, not underlines
    

    
    
    % Colors for the hyperref package
    \definecolor{urlcolor}{rgb}{0,.145,.698}
    \definecolor{linkcolor}{rgb}{.71,0.21,0.01}
    \definecolor{citecolor}{rgb}{.12,.54,.11}

    % ANSI colors
    \definecolor{ansi-black}{HTML}{3E424D}
    \definecolor{ansi-black-intense}{HTML}{282C36}
    \definecolor{ansi-red}{HTML}{E75C58}
    \definecolor{ansi-red-intense}{HTML}{B22B31}
    \definecolor{ansi-green}{HTML}{00A250}
    \definecolor{ansi-green-intense}{HTML}{007427}
    \definecolor{ansi-yellow}{HTML}{DDB62B}
    \definecolor{ansi-yellow-intense}{HTML}{B27D12}
    \definecolor{ansi-blue}{HTML}{208FFB}
    \definecolor{ansi-blue-intense}{HTML}{0065CA}
    \definecolor{ansi-magenta}{HTML}{D160C4}
    \definecolor{ansi-magenta-intense}{HTML}{A03196}
    \definecolor{ansi-cyan}{HTML}{60C6C8}
    \definecolor{ansi-cyan-intense}{HTML}{258F8F}
    \definecolor{ansi-white}{HTML}{C5C1B4}
    \definecolor{ansi-white-intense}{HTML}{A1A6B2}

    % commands and environments needed by pandoc snippets
    % extracted from the output of `pandoc -s`
    \providecommand{\tightlist}{%
      \setlength{\itemsep}{0pt}\setlength{\parskip}{0pt}}
    \DefineVerbatimEnvironment{Highlighting}{Verbatim}{commandchars=\\\{\}}
    % Add ',fontsize=\small' for more characters per line
    \newenvironment{Shaded}{}{}
    \newcommand{\KeywordTok}[1]{\textcolor[rgb]{0.00,0.44,0.13}{\textbf{{#1}}}}
    \newcommand{\DataTypeTok}[1]{\textcolor[rgb]{0.56,0.13,0.00}{{#1}}}
    \newcommand{\DecValTok}[1]{\textcolor[rgb]{0.25,0.63,0.44}{{#1}}}
    \newcommand{\BaseNTok}[1]{\textcolor[rgb]{0.25,0.63,0.44}{{#1}}}
    \newcommand{\FloatTok}[1]{\textcolor[rgb]{0.25,0.63,0.44}{{#1}}}
    \newcommand{\CharTok}[1]{\textcolor[rgb]{0.25,0.44,0.63}{{#1}}}
    \newcommand{\StringTok}[1]{\textcolor[rgb]{0.25,0.44,0.63}{{#1}}}
    \newcommand{\CommentTok}[1]{\textcolor[rgb]{0.38,0.63,0.69}{\textit{{#1}}}}
    \newcommand{\OtherTok}[1]{\textcolor[rgb]{0.00,0.44,0.13}{{#1}}}
    \newcommand{\AlertTok}[1]{\textcolor[rgb]{1.00,0.00,0.00}{\textbf{{#1}}}}
    \newcommand{\FunctionTok}[1]{\textcolor[rgb]{0.02,0.16,0.49}{{#1}}}
    \newcommand{\RegionMarkerTok}[1]{{#1}}
    \newcommand{\ErrorTok}[1]{\textcolor[rgb]{1.00,0.00,0.00}{\textbf{{#1}}}}
    \newcommand{\NormalTok}[1]{{#1}}
    
    % Additional commands for more recent versions of Pandoc
    \newcommand{\ConstantTok}[1]{\textcolor[rgb]{0.53,0.00,0.00}{{#1}}}
    \newcommand{\SpecialCharTok}[1]{\textcolor[rgb]{0.25,0.44,0.63}{{#1}}}
    \newcommand{\VerbatimStringTok}[1]{\textcolor[rgb]{0.25,0.44,0.63}{{#1}}}
    \newcommand{\SpecialStringTok}[1]{\textcolor[rgb]{0.73,0.40,0.53}{{#1}}}
    \newcommand{\ImportTok}[1]{{#1}}
    \newcommand{\DocumentationTok}[1]{\textcolor[rgb]{0.73,0.13,0.13}{\textit{{#1}}}}
    \newcommand{\AnnotationTok}[1]{\textcolor[rgb]{0.38,0.63,0.69}{\textbf{\textit{{#1}}}}}
    \newcommand{\CommentVarTok}[1]{\textcolor[rgb]{0.38,0.63,0.69}{\textbf{\textit{{#1}}}}}
    \newcommand{\VariableTok}[1]{\textcolor[rgb]{0.10,0.09,0.49}{{#1}}}
    \newcommand{\ControlFlowTok}[1]{\textcolor[rgb]{0.00,0.44,0.13}{\textbf{{#1}}}}
    \newcommand{\OperatorTok}[1]{\textcolor[rgb]{0.40,0.40,0.40}{{#1}}}
    \newcommand{\BuiltInTok}[1]{{#1}}
    \newcommand{\ExtensionTok}[1]{{#1}}
    \newcommand{\PreprocessorTok}[1]{\textcolor[rgb]{0.74,0.48,0.00}{{#1}}}
    \newcommand{\AttributeTok}[1]{\textcolor[rgb]{0.49,0.56,0.16}{{#1}}}
    \newcommand{\InformationTok}[1]{\textcolor[rgb]{0.38,0.63,0.69}{\textbf{\textit{{#1}}}}}
    \newcommand{\WarningTok}[1]{\textcolor[rgb]{0.38,0.63,0.69}{\textbf{\textit{{#1}}}}}
    
    
    % Define a nice break command that doesn't care if a line doesn't already
    % exist.
    \def\br{\hspace*{\fill} \\* }
    % Math Jax compatability definitions
    \def\gt{>}
    \def\lt{<}
    % Document parameters
    \title{mappers}
    
    
    

    % Pygments definitions
    
\makeatletter
\def\PY@reset{\let\PY@it=\relax \let\PY@bf=\relax%
    \let\PY@ul=\relax \let\PY@tc=\relax%
    \let\PY@bc=\relax \let\PY@ff=\relax}
\def\PY@tok#1{\csname PY@tok@#1\endcsname}
\def\PY@toks#1+{\ifx\relax#1\empty\else%
    \PY@tok{#1}\expandafter\PY@toks\fi}
\def\PY@do#1{\PY@bc{\PY@tc{\PY@ul{%
    \PY@it{\PY@bf{\PY@ff{#1}}}}}}}
\def\PY#1#2{\PY@reset\PY@toks#1+\relax+\PY@do{#2}}

\expandafter\def\csname PY@tok@w\endcsname{\def\PY@tc##1{\textcolor[rgb]{0.73,0.73,0.73}{##1}}}
\expandafter\def\csname PY@tok@c\endcsname{\let\PY@it=\textit\def\PY@tc##1{\textcolor[rgb]{0.25,0.50,0.50}{##1}}}
\expandafter\def\csname PY@tok@cp\endcsname{\def\PY@tc##1{\textcolor[rgb]{0.74,0.48,0.00}{##1}}}
\expandafter\def\csname PY@tok@k\endcsname{\let\PY@bf=\textbf\def\PY@tc##1{\textcolor[rgb]{0.00,0.50,0.00}{##1}}}
\expandafter\def\csname PY@tok@kp\endcsname{\def\PY@tc##1{\textcolor[rgb]{0.00,0.50,0.00}{##1}}}
\expandafter\def\csname PY@tok@kt\endcsname{\def\PY@tc##1{\textcolor[rgb]{0.69,0.00,0.25}{##1}}}
\expandafter\def\csname PY@tok@o\endcsname{\def\PY@tc##1{\textcolor[rgb]{0.40,0.40,0.40}{##1}}}
\expandafter\def\csname PY@tok@ow\endcsname{\let\PY@bf=\textbf\def\PY@tc##1{\textcolor[rgb]{0.67,0.13,1.00}{##1}}}
\expandafter\def\csname PY@tok@nb\endcsname{\def\PY@tc##1{\textcolor[rgb]{0.00,0.50,0.00}{##1}}}
\expandafter\def\csname PY@tok@nf\endcsname{\def\PY@tc##1{\textcolor[rgb]{0.00,0.00,1.00}{##1}}}
\expandafter\def\csname PY@tok@nc\endcsname{\let\PY@bf=\textbf\def\PY@tc##1{\textcolor[rgb]{0.00,0.00,1.00}{##1}}}
\expandafter\def\csname PY@tok@nn\endcsname{\let\PY@bf=\textbf\def\PY@tc##1{\textcolor[rgb]{0.00,0.00,1.00}{##1}}}
\expandafter\def\csname PY@tok@ne\endcsname{\let\PY@bf=\textbf\def\PY@tc##1{\textcolor[rgb]{0.82,0.25,0.23}{##1}}}
\expandafter\def\csname PY@tok@nv\endcsname{\def\PY@tc##1{\textcolor[rgb]{0.10,0.09,0.49}{##1}}}
\expandafter\def\csname PY@tok@no\endcsname{\def\PY@tc##1{\textcolor[rgb]{0.53,0.00,0.00}{##1}}}
\expandafter\def\csname PY@tok@nl\endcsname{\def\PY@tc##1{\textcolor[rgb]{0.63,0.63,0.00}{##1}}}
\expandafter\def\csname PY@tok@ni\endcsname{\let\PY@bf=\textbf\def\PY@tc##1{\textcolor[rgb]{0.60,0.60,0.60}{##1}}}
\expandafter\def\csname PY@tok@na\endcsname{\def\PY@tc##1{\textcolor[rgb]{0.49,0.56,0.16}{##1}}}
\expandafter\def\csname PY@tok@nt\endcsname{\let\PY@bf=\textbf\def\PY@tc##1{\textcolor[rgb]{0.00,0.50,0.00}{##1}}}
\expandafter\def\csname PY@tok@nd\endcsname{\def\PY@tc##1{\textcolor[rgb]{0.67,0.13,1.00}{##1}}}
\expandafter\def\csname PY@tok@s\endcsname{\def\PY@tc##1{\textcolor[rgb]{0.73,0.13,0.13}{##1}}}
\expandafter\def\csname PY@tok@sd\endcsname{\let\PY@it=\textit\def\PY@tc##1{\textcolor[rgb]{0.73,0.13,0.13}{##1}}}
\expandafter\def\csname PY@tok@si\endcsname{\let\PY@bf=\textbf\def\PY@tc##1{\textcolor[rgb]{0.73,0.40,0.53}{##1}}}
\expandafter\def\csname PY@tok@se\endcsname{\let\PY@bf=\textbf\def\PY@tc##1{\textcolor[rgb]{0.73,0.40,0.13}{##1}}}
\expandafter\def\csname PY@tok@sr\endcsname{\def\PY@tc##1{\textcolor[rgb]{0.73,0.40,0.53}{##1}}}
\expandafter\def\csname PY@tok@ss\endcsname{\def\PY@tc##1{\textcolor[rgb]{0.10,0.09,0.49}{##1}}}
\expandafter\def\csname PY@tok@sx\endcsname{\def\PY@tc##1{\textcolor[rgb]{0.00,0.50,0.00}{##1}}}
\expandafter\def\csname PY@tok@m\endcsname{\def\PY@tc##1{\textcolor[rgb]{0.40,0.40,0.40}{##1}}}
\expandafter\def\csname PY@tok@gh\endcsname{\let\PY@bf=\textbf\def\PY@tc##1{\textcolor[rgb]{0.00,0.00,0.50}{##1}}}
\expandafter\def\csname PY@tok@gu\endcsname{\let\PY@bf=\textbf\def\PY@tc##1{\textcolor[rgb]{0.50,0.00,0.50}{##1}}}
\expandafter\def\csname PY@tok@gd\endcsname{\def\PY@tc##1{\textcolor[rgb]{0.63,0.00,0.00}{##1}}}
\expandafter\def\csname PY@tok@gi\endcsname{\def\PY@tc##1{\textcolor[rgb]{0.00,0.63,0.00}{##1}}}
\expandafter\def\csname PY@tok@gr\endcsname{\def\PY@tc##1{\textcolor[rgb]{1.00,0.00,0.00}{##1}}}
\expandafter\def\csname PY@tok@ge\endcsname{\let\PY@it=\textit}
\expandafter\def\csname PY@tok@gs\endcsname{\let\PY@bf=\textbf}
\expandafter\def\csname PY@tok@gp\endcsname{\let\PY@bf=\textbf\def\PY@tc##1{\textcolor[rgb]{0.00,0.00,0.50}{##1}}}
\expandafter\def\csname PY@tok@go\endcsname{\def\PY@tc##1{\textcolor[rgb]{0.53,0.53,0.53}{##1}}}
\expandafter\def\csname PY@tok@gt\endcsname{\def\PY@tc##1{\textcolor[rgb]{0.00,0.27,0.87}{##1}}}
\expandafter\def\csname PY@tok@err\endcsname{\def\PY@bc##1{\setlength{\fboxsep}{0pt}\fcolorbox[rgb]{1.00,0.00,0.00}{1,1,1}{\strut ##1}}}
\expandafter\def\csname PY@tok@kc\endcsname{\let\PY@bf=\textbf\def\PY@tc##1{\textcolor[rgb]{0.00,0.50,0.00}{##1}}}
\expandafter\def\csname PY@tok@kd\endcsname{\let\PY@bf=\textbf\def\PY@tc##1{\textcolor[rgb]{0.00,0.50,0.00}{##1}}}
\expandafter\def\csname PY@tok@kn\endcsname{\let\PY@bf=\textbf\def\PY@tc##1{\textcolor[rgb]{0.00,0.50,0.00}{##1}}}
\expandafter\def\csname PY@tok@kr\endcsname{\let\PY@bf=\textbf\def\PY@tc##1{\textcolor[rgb]{0.00,0.50,0.00}{##1}}}
\expandafter\def\csname PY@tok@bp\endcsname{\def\PY@tc##1{\textcolor[rgb]{0.00,0.50,0.00}{##1}}}
\expandafter\def\csname PY@tok@fm\endcsname{\def\PY@tc##1{\textcolor[rgb]{0.00,0.00,1.00}{##1}}}
\expandafter\def\csname PY@tok@vc\endcsname{\def\PY@tc##1{\textcolor[rgb]{0.10,0.09,0.49}{##1}}}
\expandafter\def\csname PY@tok@vg\endcsname{\def\PY@tc##1{\textcolor[rgb]{0.10,0.09,0.49}{##1}}}
\expandafter\def\csname PY@tok@vi\endcsname{\def\PY@tc##1{\textcolor[rgb]{0.10,0.09,0.49}{##1}}}
\expandafter\def\csname PY@tok@vm\endcsname{\def\PY@tc##1{\textcolor[rgb]{0.10,0.09,0.49}{##1}}}
\expandafter\def\csname PY@tok@sa\endcsname{\def\PY@tc##1{\textcolor[rgb]{0.73,0.13,0.13}{##1}}}
\expandafter\def\csname PY@tok@sb\endcsname{\def\PY@tc##1{\textcolor[rgb]{0.73,0.13,0.13}{##1}}}
\expandafter\def\csname PY@tok@sc\endcsname{\def\PY@tc##1{\textcolor[rgb]{0.73,0.13,0.13}{##1}}}
\expandafter\def\csname PY@tok@dl\endcsname{\def\PY@tc##1{\textcolor[rgb]{0.73,0.13,0.13}{##1}}}
\expandafter\def\csname PY@tok@s2\endcsname{\def\PY@tc##1{\textcolor[rgb]{0.73,0.13,0.13}{##1}}}
\expandafter\def\csname PY@tok@sh\endcsname{\def\PY@tc##1{\textcolor[rgb]{0.73,0.13,0.13}{##1}}}
\expandafter\def\csname PY@tok@s1\endcsname{\def\PY@tc##1{\textcolor[rgb]{0.73,0.13,0.13}{##1}}}
\expandafter\def\csname PY@tok@mb\endcsname{\def\PY@tc##1{\textcolor[rgb]{0.40,0.40,0.40}{##1}}}
\expandafter\def\csname PY@tok@mf\endcsname{\def\PY@tc##1{\textcolor[rgb]{0.40,0.40,0.40}{##1}}}
\expandafter\def\csname PY@tok@mh\endcsname{\def\PY@tc##1{\textcolor[rgb]{0.40,0.40,0.40}{##1}}}
\expandafter\def\csname PY@tok@mi\endcsname{\def\PY@tc##1{\textcolor[rgb]{0.40,0.40,0.40}{##1}}}
\expandafter\def\csname PY@tok@il\endcsname{\def\PY@tc##1{\textcolor[rgb]{0.40,0.40,0.40}{##1}}}
\expandafter\def\csname PY@tok@mo\endcsname{\def\PY@tc##1{\textcolor[rgb]{0.40,0.40,0.40}{##1}}}
\expandafter\def\csname PY@tok@ch\endcsname{\let\PY@it=\textit\def\PY@tc##1{\textcolor[rgb]{0.25,0.50,0.50}{##1}}}
\expandafter\def\csname PY@tok@cm\endcsname{\let\PY@it=\textit\def\PY@tc##1{\textcolor[rgb]{0.25,0.50,0.50}{##1}}}
\expandafter\def\csname PY@tok@cpf\endcsname{\let\PY@it=\textit\def\PY@tc##1{\textcolor[rgb]{0.25,0.50,0.50}{##1}}}
\expandafter\def\csname PY@tok@c1\endcsname{\let\PY@it=\textit\def\PY@tc##1{\textcolor[rgb]{0.25,0.50,0.50}{##1}}}
\expandafter\def\csname PY@tok@cs\endcsname{\let\PY@it=\textit\def\PY@tc##1{\textcolor[rgb]{0.25,0.50,0.50}{##1}}}

\def\PYZbs{\char`\\}
\def\PYZus{\char`\_}
\def\PYZob{\char`\{}
\def\PYZcb{\char`\}}
\def\PYZca{\char`\^}
\def\PYZam{\char`\&}
\def\PYZlt{\char`\<}
\def\PYZgt{\char`\>}
\def\PYZsh{\char`\#}
\def\PYZpc{\char`\%}
\def\PYZdl{\char`\$}
\def\PYZhy{\char`\-}
\def\PYZsq{\char`\'}
\def\PYZdq{\char`\"}
\def\PYZti{\char`\~}
% for compatibility with earlier versions
\def\PYZat{@}
\def\PYZlb{[}
\def\PYZrb{]}
\makeatother


    % Exact colors from NB
    \definecolor{incolor}{rgb}{0.0, 0.0, 0.5}
    \definecolor{outcolor}{rgb}{0.545, 0.0, 0.0}



    
    % Prevent overflowing lines due to hard-to-break entities
    \sloppy 
    % Setup hyperref package
    \hypersetup{
      breaklinks=true,  % so long urls are correctly broken across lines
      colorlinks=true,
      urlcolor=urlcolor,
      linkcolor=linkcolor,
      citecolor=citecolor,
      }
    % Slightly bigger margins than the latex defaults
    
    \geometry{verbose,tmargin=1in,bmargin=1in,lmargin=1in,rmargin=1in}
    
    

    \begin{document}
    
    
    \maketitle
    
    

    
    

    \section{Mapping values between grid
elements}\label{mapping-values-between-grid-elements}

 (Note: for instructions on how to run an interactive iPython notebook,
click here: https://github.com/landlab/tutorials/blob/master/README.md
For the unexpanded version to download and run, click here:
https://nbviewer.jupyter.org/github/landlab/tutorials/blob/master/mappers/mappers\_unexpanded.ipynb
For more Landlab tutorials, click here:
https://github.com/landlab/landlab/wiki/Tutorials)

Imagine that you're using Landlab to write a model of shallow water flow
over terrain. A natural approach is to place your scalar fields, such as
water depth, at the nodes. You then place your vector fields, such as
water surface gradient, flow velocity, and discharge, at the links. But
your velocity depends on both slope and depth, which means you need to
know the depth at the links too. How do you do this?

This tutorial introduces \emph{mappers}: grid functions that map
quantities defined on one set of elements (such as nodes) onto another
set of elements (such as links). As you'll see, there are a variety of
mappers available.

    \subsection{Mapping from nodes to
links}\label{mapping-from-nodes-to-links}

For the sake of example, we'll start with a simple 3-row by 4-column
raster grid. The grid will contain a scalar field called
\texttt{water\_\_depth}, abbreviated \texttt{h}. We'll populate it with
some example values, as follows:

    \begin{Verbatim}[commandchars=\\\{\}]
{\color{incolor}In [{\color{incolor}1}]:} \PY{k+kn}{from} \PY{n+nn}{\PYZus{}\PYZus{}future\PYZus{}\PYZus{}} \PY{k}{import} \PY{n}{print\PYZus{}function}
\end{Verbatim}


    \begin{Verbatim}[commandchars=\\\{\}]
{\color{incolor}In [{\color{incolor}2}]:} \PY{k+kn}{from} \PY{n+nn}{\PYZus{}\PYZus{}future\PYZus{}\PYZus{}} \PY{k}{import} \PY{n}{print\PYZus{}function}
        \PY{k+kn}{from} \PY{n+nn}{landlab} \PY{k}{import} \PY{n}{RasterModelGrid}
        \PY{k+kn}{import} \PY{n+nn}{numpy} \PY{k}{as} \PY{n+nn}{np}
        \PY{n}{mg} \PY{o}{=} \PY{n}{RasterModelGrid}\PY{p}{(}\PY{p}{(}\PY{l+m+mi}{3}\PY{p}{,} \PY{l+m+mi}{4}\PY{p}{)}\PY{p}{,} \PY{l+m+mf}{100.0}\PY{p}{)}
        \PY{n}{h} \PY{o}{=} \PY{n}{mg}\PY{o}{.}\PY{n}{add\PYZus{}zeros}\PY{p}{(}\PY{l+s+s1}{\PYZsq{}}\PY{l+s+s1}{node}\PY{l+s+s1}{\PYZsq{}}\PY{p}{,} \PY{l+s+s1}{\PYZsq{}}\PY{l+s+s1}{surface\PYZus{}water\PYZus{}\PYZus{}depth}\PY{l+s+s1}{\PYZsq{}}\PY{p}{)}
        \PY{n}{h}\PY{p}{[}\PY{p}{:}\PY{p}{]} \PY{o}{=} \PY{l+m+mi}{7} \PY{o}{\PYZhy{}} \PY{n}{np}\PY{o}{.}\PY{n}{abs}\PY{p}{(}\PY{l+m+mi}{6} \PY{o}{\PYZhy{}} \PY{n}{np}\PY{o}{.}\PY{n}{arange}\PY{p}{(}\PY{l+m+mi}{12}\PY{p}{)}\PY{p}{)}
\end{Verbatim}


    For the sake of visualizing values at nodes on our grid, we'll define a
handy little function:

    \begin{Verbatim}[commandchars=\\\{\}]
{\color{incolor}In [{\color{incolor}3}]:} \PY{k}{def} \PY{n+nf}{show\PYZus{}node\PYZus{}values}\PY{p}{(}\PY{n}{mg}\PY{p}{,} \PY{n}{u}\PY{p}{)}\PY{p}{:}
            \PY{k}{for} \PY{n}{r} \PY{o+ow}{in} \PY{n+nb}{range}\PY{p}{(}\PY{n}{mg}\PY{o}{.}\PY{n}{number\PYZus{}of\PYZus{}node\PYZus{}rows} \PY{o}{\PYZhy{}} \PY{l+m+mi}{1}\PY{p}{,} \PY{o}{\PYZhy{}}\PY{l+m+mi}{1}\PY{p}{,} \PY{o}{\PYZhy{}}\PY{l+m+mi}{1}\PY{p}{)}\PY{p}{:}
                \PY{k}{for} \PY{n}{c} \PY{o+ow}{in} \PY{n+nb}{range}\PY{p}{(}\PY{n}{mg}\PY{o}{.}\PY{n}{number\PYZus{}of\PYZus{}node\PYZus{}columns}\PY{p}{)}\PY{p}{:}
                    \PY{n+nb}{print}\PY{p}{(}\PY{n+nb}{int}\PY{p}{(}\PY{n}{u}\PY{p}{[}\PY{n}{c} \PY{o}{+} \PY{p}{(}\PY{n}{mg}\PY{o}{.}\PY{n}{number\PYZus{}of\PYZus{}node\PYZus{}columns} \PY{o}{*} \PY{n}{r}\PY{p}{)}\PY{p}{]}\PY{p}{)}\PY{p}{,} \PY{n}{end}\PY{o}{=}\PY{l+s+s1}{\PYZsq{}}\PY{l+s+s1}{ }\PY{l+s+s1}{\PYZsq{}}\PY{p}{)}
                \PY{n+nb}{print}\PY{p}{(}\PY{p}{)}
\end{Verbatim}


    \begin{Verbatim}[commandchars=\\\{\}]
{\color{incolor}In [{\color{incolor}4}]:} \PY{n}{show\PYZus{}node\PYZus{}values}\PY{p}{(}\PY{n}{mg}\PY{p}{,} \PY{n}{h}\PY{p}{)}
\end{Verbatim}


    \begin{Verbatim}[commandchars=\\\{\}]
5 4 3 2 
5 6 7 6 
1 2 3 4 

    \end{Verbatim}

    Let's review the numbering of nodes and links. The lines below will
print a list that shows, for each link ID, the IDs of the nodes at the
link's tail and head:

    \begin{Verbatim}[commandchars=\\\{\}]
{\color{incolor}In [{\color{incolor}5}]:} \PY{k}{for} \PY{n}{i} \PY{o+ow}{in} \PY{n+nb}{range}\PY{p}{(}\PY{n}{mg}\PY{o}{.}\PY{n}{number\PYZus{}of\PYZus{}links}\PY{p}{)}\PY{p}{:}
            \PY{n+nb}{print}\PY{p}{(}\PY{n}{i}\PY{p}{,} \PY{n}{mg}\PY{o}{.}\PY{n}{node\PYZus{}at\PYZus{}link\PYZus{}tail}\PY{p}{[}\PY{n}{i}\PY{p}{]}\PY{p}{,} \PY{n}{mg}\PY{o}{.}\PY{n}{node\PYZus{}at\PYZus{}link\PYZus{}head}\PY{p}{[}\PY{n}{i}\PY{p}{]}\PY{p}{)}
\end{Verbatim}


    \begin{Verbatim}[commandchars=\\\{\}]
0 0 1
1 1 2
2 2 3
3 0 4
4 1 5
5 2 6
6 3 7
7 4 5
8 5 6
9 6 7
10 4 8
11 5 9
12 6 10
13 7 11
14 8 9
15 9 10
16 10 11

    \end{Verbatim}

    \subsubsection{Finding the mean value between two nodes on a
link}\label{finding-the-mean-value-between-two-nodes-on-a-link}

Suppose we want to have a \emph{link-based} array, called
\emph{h\_edge}, that contains water depth at locations between adjacent
pairs of nodes. For each link, we'll simply take the average of the
depth at the link's two nodes. To accomplish this, we can use the
\texttt{map\_mean\_of\_link\_nodes\_to\_link} grid method. At link 8,
for example, we'll average the \emph{h} values at nodes 5 and 6, which
should give us a depth of (6 + 7) / 2 = 6.5:

    \begin{Verbatim}[commandchars=\\\{\}]
{\color{incolor}In [{\color{incolor}6}]:} \PY{n}{h\PYZus{}edge} \PY{o}{=} \PY{n}{mg}\PY{o}{.}\PY{n}{map\PYZus{}mean\PYZus{}of\PYZus{}link\PYZus{}nodes\PYZus{}to\PYZus{}link}\PY{p}{(}\PY{l+s+s1}{\PYZsq{}}\PY{l+s+s1}{surface\PYZus{}water\PYZus{}\PYZus{}depth}\PY{l+s+s1}{\PYZsq{}}\PY{p}{)}
        \PY{k}{for} \PY{n}{i} \PY{o+ow}{in} \PY{n+nb}{range}\PY{p}{(}\PY{n}{mg}\PY{o}{.}\PY{n}{number\PYZus{}of\PYZus{}links}\PY{p}{)}\PY{p}{:}
            \PY{n+nb}{print}\PY{p}{(}\PY{n}{i}\PY{p}{,} \PY{n}{h\PYZus{}edge}\PY{p}{[}\PY{n}{i}\PY{p}{]}\PY{p}{)}
\end{Verbatim}


    \begin{Verbatim}[commandchars=\\\{\}]
0 1.5
1 2.5
2 3.5
3 3.0
4 4.0
5 5.0
6 5.0
7 5.5
8 6.5
9 6.5
10 5.0
11 5.0
12 5.0
13 4.0
14 4.5
15 3.5
16 2.5

    \end{Verbatim}

    \subsubsection{What's in a name?}\label{whats-in-a-name}

The mapping function has a long name, which is designed to make it as
clear as possible to understand what the function does. All the mappers
start with the verb \emph{map}. Then the relationship is given; in this
case, we are looking at the \emph{mean}. Then the elements from which a
quantity is being mapped: we are taking values from \emph{link nodes}.
Finally, the element to which the new values apply: \emph{link}.

\subsubsection{Mapping minimum or maximum
values}\label{mapping-minimum-or-maximum-values}

We can also map the minimum value of \emph{h}:

    \begin{Verbatim}[commandchars=\\\{\}]
{\color{incolor}In [{\color{incolor}7}]:} \PY{n}{h\PYZus{}edge} \PY{o}{=} \PY{n}{mg}\PY{o}{.}\PY{n}{map\PYZus{}min\PYZus{}of\PYZus{}link\PYZus{}nodes\PYZus{}to\PYZus{}link}\PY{p}{(}\PY{l+s+s1}{\PYZsq{}}\PY{l+s+s1}{surface\PYZus{}water\PYZus{}\PYZus{}depth}\PY{l+s+s1}{\PYZsq{}}\PY{p}{)}
        \PY{k}{for} \PY{n}{i} \PY{o+ow}{in} \PY{n+nb}{range}\PY{p}{(}\PY{n}{mg}\PY{o}{.}\PY{n}{number\PYZus{}of\PYZus{}links}\PY{p}{)}\PY{p}{:}
            \PY{n+nb}{print}\PY{p}{(}\PY{n}{i}\PY{p}{,} \PY{n}{h\PYZus{}edge}\PY{p}{[}\PY{n}{i}\PY{p}{]}\PY{p}{)}
\end{Verbatim}


    \begin{Verbatim}[commandchars=\\\{\}]
0 1.0
1 2.0
2 3.0
3 1.0
4 2.0
5 3.0
6 4.0
7 5.0
8 6.0
9 6.0
10 5.0
11 4.0
12 3.0
13 2.0
14 4.0
15 3.0
16 2.0

    \end{Verbatim}

    ... or the maximum:

    \begin{Verbatim}[commandchars=\\\{\}]
{\color{incolor}In [{\color{incolor}8}]:} \PY{n}{h\PYZus{}edge} \PY{o}{=} \PY{n}{mg}\PY{o}{.}\PY{n}{map\PYZus{}max\PYZus{}of\PYZus{}link\PYZus{}nodes\PYZus{}to\PYZus{}link}\PY{p}{(}\PY{l+s+s1}{\PYZsq{}}\PY{l+s+s1}{surface\PYZus{}water\PYZus{}\PYZus{}depth}\PY{l+s+s1}{\PYZsq{}}\PY{p}{)}
        \PY{k}{for} \PY{n}{i} \PY{o+ow}{in} \PY{n+nb}{range}\PY{p}{(}\PY{n}{mg}\PY{o}{.}\PY{n}{number\PYZus{}of\PYZus{}links}\PY{p}{)}\PY{p}{:}
            \PY{n+nb}{print}\PY{p}{(}\PY{n}{i}\PY{p}{,} \PY{n}{h\PYZus{}edge}\PY{p}{[}\PY{n}{i}\PY{p}{]}\PY{p}{)}
\end{Verbatim}


    \begin{Verbatim}[commandchars=\\\{\}]
0 2.0
1 3.0
2 4.0
3 5.0
4 6.0
5 7.0
6 6.0
7 6.0
8 7.0
9 7.0
10 5.0
11 6.0
12 7.0
13 6.0
14 5.0
15 4.0
16 3.0

    \end{Verbatim}

    \subsubsection{Upwind and downwind}\label{upwind-and-downwind}

Numerical schemes often use \emph{upwind differencing} or \emph{downwind
differencing}. For example, finite difference schemes for equations that
include advection may use "upwind" rather than centered differences, in
which a scalar quantity (our \emph{h} for example) is taken from
whichever side is upstream in the flow field.

How do we know the flow direction? If the flow is driven by the gradient
in some scalar field, such as pressure or elevation, one approach is to
look at the values of this scalar on either end of the link: the end
with the higher value is upwind, and the end with the lower value is
downwind.

Suppose for example that our water flow is driven by the water-surface
slope (which is often a good approximation for the \emph{energy slope},
though it omits the kinetic energy). Let's define a bed-surface
elevation field \emph{z}:

    \begin{Verbatim}[commandchars=\\\{\}]
{\color{incolor}In [{\color{incolor}9}]:} \PY{n}{z} \PY{o}{=} \PY{n}{mg}\PY{o}{.}\PY{n}{add\PYZus{}zeros}\PY{p}{(}\PY{l+s+s1}{\PYZsq{}}\PY{l+s+s1}{node}\PY{l+s+s1}{\PYZsq{}}\PY{p}{,} \PY{l+s+s1}{\PYZsq{}}\PY{l+s+s1}{topographic\PYZus{}\PYZus{}elevation}\PY{l+s+s1}{\PYZsq{}}\PY{p}{)}
        \PY{n}{z}\PY{p}{[}\PY{p}{:}\PY{p}{]} \PY{o}{=} \PY{l+m+mi}{16} \PY{o}{\PYZhy{}} \PY{n}{np}\PY{o}{.}\PY{n}{abs}\PY{p}{(}\PY{l+m+mi}{7} \PY{o}{\PYZhy{}} \PY{n}{np}\PY{o}{.}\PY{n}{arange}\PY{p}{(}\PY{l+m+mi}{12}\PY{p}{)}\PY{p}{)}
        \PY{n}{show\PYZus{}node\PYZus{}values}\PY{p}{(}\PY{n}{mg}\PY{p}{,} \PY{n}{z}\PY{p}{)}
\end{Verbatim}


    \begin{Verbatim}[commandchars=\\\{\}]
15 14 13 12 
13 14 15 16 
9 10 11 12 

    \end{Verbatim}

    The water-surface elevation is then the sum of \emph{h} and \emph{z}:

    \begin{Verbatim}[commandchars=\\\{\}]
{\color{incolor}In [{\color{incolor}10}]:} \PY{n}{w} \PY{o}{=} \PY{n}{z} \PY{o}{+} \PY{n}{h}
         \PY{n}{show\PYZus{}node\PYZus{}values}\PY{p}{(}\PY{n}{mg}\PY{p}{,} \PY{n}{w}\PY{p}{)}
\end{Verbatim}


    \begin{Verbatim}[commandchars=\\\{\}]
20 18 16 14 
18 20 22 22 
10 12 14 16 

    \end{Verbatim}

    For every link, we can assign the value of \emph{h} from whichever end
of the link has the greater \emph{w}:

    \begin{Verbatim}[commandchars=\\\{\}]
{\color{incolor}In [{\color{incolor}11}]:} \PY{n}{h\PYZus{}edge} \PY{o}{=} \PY{n}{mg}\PY{o}{.}\PY{n}{map\PYZus{}value\PYZus{}at\PYZus{}max\PYZus{}node\PYZus{}to\PYZus{}link}\PY{p}{(}\PY{n}{w}\PY{p}{,} \PY{n}{h}\PY{p}{)}
         \PY{k}{for} \PY{n}{i} \PY{o+ow}{in} \PY{n+nb}{range}\PY{p}{(}\PY{n}{mg}\PY{o}{.}\PY{n}{number\PYZus{}of\PYZus{}links}\PY{p}{)}\PY{p}{:}
             \PY{n+nb}{print}\PY{p}{(}\PY{n}{i}\PY{p}{,} \PY{n}{h\PYZus{}edge}\PY{p}{[}\PY{n}{i}\PY{p}{]}\PY{p}{)}
\end{Verbatim}


    \begin{Verbatim}[commandchars=\\\{\}]
0 2.0
1 3.0
2 4.0
3 5.0
4 6.0
5 7.0
6 6.0
7 6.0
8 7.0
9 6.0
10 5.0
11 6.0
12 7.0
13 6.0
14 5.0
15 4.0
16 3.0

    \end{Verbatim}

    Consider the middle two nodes (5 and 6). Node 6 is higher (22 versus
20). Therefore, the link between them (link 8) should be assigned the
value of \emph{h} at node 6. This value happens to be 7.0.

Of course, we could also take the value from the \emph{lower} of the two
nodes, which gives link 8 a value of 6.0:

    \begin{Verbatim}[commandchars=\\\{\}]
{\color{incolor}In [{\color{incolor}12}]:} \PY{n}{h\PYZus{}edge} \PY{o}{=} \PY{n}{mg}\PY{o}{.}\PY{n}{map\PYZus{}value\PYZus{}at\PYZus{}min\PYZus{}node\PYZus{}to\PYZus{}link}\PY{p}{(}\PY{n}{w}\PY{p}{,} \PY{n}{h}\PY{p}{)}
         \PY{k}{for} \PY{n}{i} \PY{o+ow}{in} \PY{n+nb}{range}\PY{p}{(}\PY{n}{mg}\PY{o}{.}\PY{n}{number\PYZus{}of\PYZus{}links}\PY{p}{)}\PY{p}{:}
             \PY{n+nb}{print}\PY{p}{(}\PY{n}{i}\PY{p}{,} \PY{n}{h\PYZus{}edge}\PY{p}{[}\PY{n}{i}\PY{p}{]}\PY{p}{)}
\end{Verbatim}


    \begin{Verbatim}[commandchars=\\\{\}]
0 1.0
1 2.0
2 3.0
3 1.0
4 2.0
5 3.0
6 4.0
7 5.0
8 6.0
9 6.0
10 5.0
11 4.0
12 3.0
13 2.0
14 4.0
15 3.0
16 2.0

    \end{Verbatim}

    \subsubsection{Heads or tails?}\label{heads-or-tails}

It is also possible to map the scalar quantity at either the head node
or the tail node to the link:

    \begin{Verbatim}[commandchars=\\\{\}]
{\color{incolor}In [{\color{incolor}13}]:} \PY{n}{h\PYZus{}edge} \PY{o}{=} \PY{n}{mg}\PY{o}{.}\PY{n}{map\PYZus{}link\PYZus{}head\PYZus{}node\PYZus{}to\PYZus{}link}\PY{p}{(}\PY{l+s+s1}{\PYZsq{}}\PY{l+s+s1}{surface\PYZus{}water\PYZus{}\PYZus{}depth}\PY{l+s+s1}{\PYZsq{}}\PY{p}{)}
         \PY{k}{for} \PY{n}{i} \PY{o+ow}{in} \PY{n+nb}{range}\PY{p}{(}\PY{n}{mg}\PY{o}{.}\PY{n}{number\PYZus{}of\PYZus{}links}\PY{p}{)}\PY{p}{:}
             \PY{n+nb}{print}\PY{p}{(}\PY{n}{i}\PY{p}{,} \PY{n}{h\PYZus{}edge}\PY{p}{[}\PY{n}{i}\PY{p}{]}\PY{p}{)}
\end{Verbatim}


    \begin{Verbatim}[commandchars=\\\{\}]
0 2.0
1 3.0
2 4.0
3 5.0
4 6.0
5 7.0
6 6.0
7 6.0
8 7.0
9 6.0
10 5.0
11 4.0
12 3.0
13 2.0
14 4.0
15 3.0
16 2.0

    \end{Verbatim}

    \begin{Verbatim}[commandchars=\\\{\}]
{\color{incolor}In [{\color{incolor}14}]:} \PY{n}{h\PYZus{}edge} \PY{o}{=} \PY{n}{mg}\PY{o}{.}\PY{n}{map\PYZus{}link\PYZus{}tail\PYZus{}node\PYZus{}to\PYZus{}link}\PY{p}{(}\PY{l+s+s1}{\PYZsq{}}\PY{l+s+s1}{surface\PYZus{}water\PYZus{}\PYZus{}depth}\PY{l+s+s1}{\PYZsq{}}\PY{p}{)}
         \PY{k}{for} \PY{n}{i} \PY{o+ow}{in} \PY{n+nb}{range}\PY{p}{(}\PY{n}{mg}\PY{o}{.}\PY{n}{number\PYZus{}of\PYZus{}links}\PY{p}{)}\PY{p}{:}
             \PY{n+nb}{print}\PY{p}{(}\PY{n}{i}\PY{p}{,} \PY{n}{h\PYZus{}edge}\PY{p}{[}\PY{n}{i}\PY{p}{]}\PY{p}{)}
\end{Verbatim}


    \begin{Verbatim}[commandchars=\\\{\}]
0 1.0
1 2.0
2 3.0
3 1.0
4 2.0
5 3.0
6 4.0
7 5.0
8 6.0
9 7.0
10 5.0
11 6.0
12 7.0
13 6.0
14 5.0
15 4.0
16 3.0

    \end{Verbatim}

    \subsubsection{Simple example using centered water
depth}\label{simple-example-using-centered-water-depth}

The following implements one time-step of a linear-viscous flow model,
in which flow velocity is calculated with depth at the links is taken as
the mean of depth at the two bounding nodes. To make the flow a little
tamer, we'll have our fluid be basaltic lava instead of water, with a
dynamic viscosity of 100 Pa s.

    \begin{Verbatim}[commandchars=\\\{\}]
{\color{incolor}In [{\color{incolor}15}]:} \PY{n}{gamma} \PY{o}{=} \PY{l+m+mf}{25000.0}  \PY{c+c1}{\PYZsh{} unit weight of fluid, N/m2}
         \PY{n}{viscosity} \PY{o}{=} \PY{l+m+mf}{100.0}  \PY{c+c1}{\PYZsh{} dynamic viscosity in Pa s}
         \PY{n}{grad} \PY{o}{=} \PY{n}{mg}\PY{o}{.}\PY{n}{calc\PYZus{}grad\PYZus{}at\PYZus{}link}\PY{p}{(}\PY{n}{w}\PY{p}{)}
         \PY{n}{h\PYZus{}edge} \PY{o}{=} \PY{n}{mg}\PY{o}{.}\PY{n}{map\PYZus{}mean\PYZus{}of\PYZus{}link\PYZus{}nodes\PYZus{}to\PYZus{}link}\PY{p}{(}\PY{n}{h}\PY{p}{)}
         \PY{n}{vel} \PY{o}{=} \PY{o}{\PYZhy{}}\PY{p}{(}\PY{n}{gamma} \PY{o}{/} \PY{n}{viscosity}\PY{p}{)} \PY{o}{*} \PY{n}{h\PYZus{}edge} \PY{o}{*} \PY{n}{h\PYZus{}edge} \PY{o}{*} \PY{n}{grad}
         \PY{k}{for} \PY{n}{ln} \PY{o+ow}{in} \PY{n+nb}{range}\PY{p}{(}\PY{n}{mg}\PY{o}{.}\PY{n}{number\PYZus{}of\PYZus{}links}\PY{p}{)}\PY{p}{:}
             \PY{n+nb}{print}\PY{p}{(}\PY{n}{ln}\PY{p}{,} \PY{n}{h\PYZus{}edge}\PY{p}{[}\PY{n}{ln}\PY{p}{]}\PY{p}{,} \PY{n}{grad}\PY{p}{[}\PY{n}{ln}\PY{p}{]}\PY{p}{,} \PY{n}{vel}\PY{p}{[}\PY{n}{ln}\PY{p}{]}\PY{p}{)}
\end{Verbatim}


    \begin{Verbatim}[commandchars=\\\{\}]
0 1.5 0.02 -11.25
1 2.5 0.02 -31.25
2 3.5 0.02 -61.25
3 3.0 0.08 -180.0
4 4.0 0.08 -320.0
5 5.0 0.08 -500.0
6 5.0 0.06 -375.0
7 5.5 0.02 -151.25
8 6.5 0.02 -211.25
9 6.5 0.0 -0.0
10 5.0 0.02 -125.0
11 5.0 -0.02 125.0
12 5.0 -0.06 375.0
13 4.0 -0.08 320.0
14 4.5 -0.02 101.25
15 3.5 -0.02 61.25
16 2.5 -0.02 31.25

    \end{Verbatim}

    I'm not sure I love the idea of a 5-m thick lava flow moving at over 100
m/s! (I guess we can take some comfort from the thought that turbulence
would probably slow it down)

How different would the numerical solution be using an upwind scheme for
flow depth? Let's find out:

    \begin{Verbatim}[commandchars=\\\{\}]
{\color{incolor}In [{\color{incolor}16}]:} \PY{n}{h\PYZus{}edge} \PY{o}{=} \PY{n}{mg}\PY{o}{.}\PY{n}{map\PYZus{}value\PYZus{}at\PYZus{}max\PYZus{}node\PYZus{}to\PYZus{}link}\PY{p}{(}\PY{n}{w}\PY{p}{,} \PY{n}{h}\PY{p}{)}
         \PY{n}{vel} \PY{o}{=} \PY{o}{\PYZhy{}}\PY{p}{(}\PY{n}{gamma} \PY{o}{/} \PY{n}{viscosity}\PY{p}{)} \PY{o}{*} \PY{n}{h\PYZus{}edge} \PY{o}{*} \PY{n}{h\PYZus{}edge} \PY{o}{*} \PY{n}{grad}
         \PY{k}{for} \PY{n}{ln} \PY{o+ow}{in} \PY{n+nb}{range}\PY{p}{(}\PY{n}{mg}\PY{o}{.}\PY{n}{number\PYZus{}of\PYZus{}links}\PY{p}{)}\PY{p}{:}
             \PY{n+nb}{print}\PY{p}{(}\PY{n}{ln}\PY{p}{,} \PY{n}{h\PYZus{}edge}\PY{p}{[}\PY{n}{ln}\PY{p}{]}\PY{p}{,} \PY{n}{grad}\PY{p}{[}\PY{n}{ln}\PY{p}{]}\PY{p}{,} \PY{n}{vel}\PY{p}{[}\PY{n}{ln}\PY{p}{]}\PY{p}{)}
\end{Verbatim}


    \begin{Verbatim}[commandchars=\\\{\}]
0 2.0 0.02 -20.0
1 3.0 0.02 -45.0
2 4.0 0.02 -80.0
3 5.0 0.08 -500.0
4 6.0 0.08 -720.0
5 7.0 0.08 -980.0
6 6.0 0.06 -540.0
7 6.0 0.02 -180.0
8 7.0 0.02 -245.0
9 6.0 0.0 -0.0
10 5.0 0.02 -125.0
11 6.0 -0.02 180.0
12 7.0 -0.06 735.0
13 6.0 -0.08 720.0
14 5.0 -0.02 125.0
15 4.0 -0.02 80.0
16 3.0 -0.02 45.0

    \end{Verbatim}

    Still pretty scary.

In any event, this example illustrates how you can use Landlab's mapping
functions to build mass-conservation models in which the flow rate
depends on a gradient and a scalar, both of which can be evaluated at
links.


    % Add a bibliography block to the postdoc
    
    
    
    \end{document}
